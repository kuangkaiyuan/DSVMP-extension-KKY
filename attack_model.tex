\section{The Attack Model}\label{sec:attack}
\subsection{Attacking methods}
The classical approach to reverse engineer a VM-protected program typically
follows three steps~\cite{10falliere2009inside,17rolles2009unpacking}, described as follows.
The first step is to understand how each components of a VM interpreter works. 
To do so, the attacker needs to locate
these components and analyze how the dispatcher schedules bytecode instructions.
The second step is to understand how each bytecode is mapped to machine code
and work out the semantics of the bytecode instructions.
The third step is to use knowledge obtained in the first two steps to recover the
logical of the target program, through removing the redundant information
and restoring a program that is similar to the original program.


To perform such an attack, a significant portion of the time will have to spend in analyzing the working 
mechanism of the VM.
The problem is that a skilled attacker could  reuse knowledge
gathered from parts of the program to analyze other protected regions of
the same program or other applications protected using the same VM scheme and bytecode instructions.

\subsection{Threat model} 
Our attack model assumes that the attacker has the necessary tools and skills to implement the above attacks.
We assume the adversary holds an executable binary of the
target software and can run the program in a control
environment~\cite{11collberg2002watermarking}. We also assume the adversary
can access content stored in memory and registers , trace program
instructions, and modify the program instructions and control flows.
All these can be achieved using advance tools like ``IDA"~\cite{14Idapro}, ``Ollydbg"~\cite{15Ollydbg} and
``Sysinternals suite"~\cite{16Sysinternalssuite}. The aim of the adversary is
to completely reverse the internal implementation of the target program.
Our goal is to increase the difficulties in terms of time and efforts for an adversary to
reverse the target program implementation protected using VM-based code obfuscation. 