\section{Introduction}
Unauthorized code analysis and modification based on reverse engineering is a major concern for software companies.
Such attacks can lead to a number of undesired outcomes,
including cheating in online games, unauthorized use of software, pirated pay-tv etc.
Industry is looking for solutions for this issue to deter reverse engineering of software systems.
By making sensitive code difficult to be traced or analyzed, code obfuscation is a potential solution for the problem.

Code virtualization based on a virtual machine (VM) is emerging as a
promising way for implementing code
obfuscation~\cite{1Themida,2CV,3Vmprotect,5fang2011multi,6ming2011software,7wang2014tdvmp,8wang2013nislvmp}.
The underlying principal of VM-based protection is to replace the program
instructions with virtual bytecodes which attackers are unfamiliar with.
These virtual bytecodes will then be translated into native machine code at
runtime to be executed on the underlying hardware platform. Using a VM-based
scheme, the execution path of the obfuscated code is controlled by a virtual
instruction scheduler. A typical scheduler consists of two components: a
\emph{dispatcher} that determines which bytecode is ready for execution, and
a set of \emph{bytecode handlers} that first decode the bytecode into virtual
instructions and then translate the virtual instructions to native
machine code. This process replaces the original program instructions with
bespoke bytecode, allowing developers to conceal the purpose or logic of
sensitive code regions.

Prior work on VM-based software protection primarily focuses on making a
single set of bytecodes more complicate, and uses one single virtual
instruction scheduler. This is based on the assumption that the scheduler and
the bytecode instruction set are difficult to be analyzed in most practical
runtime environments. However, research has shown that this is an unreliable
assumption~\cite{10falliere2009inside} in certain scenarios where an
adversary can easily reuse knowledge obtained from other applications
protected with the same scheme to preform reverse engineering (referred as
\emph{cumulative attacks} in this paper). To protect software against
cumulative attacks, it is important to have a certain degree of non-deterministic
and diversity during program execution~\cite{4collberg}.

This paper presents \DSVMP (\emph{dynamic scheduling for VM-based code
protection}), a novel VM-based code protection scheme to address the problem
of cumulative attacks. Our key insight is that it will be more difficult for
the attacker to track the application logic if sensitive code regions behave differently
in different runs. \DSVMP achieves this by introducing rich non-determinism and
diversity into program execution. To do so, it exploits a flexible,
multi-dispatched scheme for code scheduling and interpretation. Unlike prior
work where a program always follows a single, fixed execution path for the
same input across different runs, the \DSVMP scheduler directs the program to
execute a randomly selected path for each protected code region. As a
result, the program follows different execution paths in different runs and
exposes an non-deterministic behavior. Our carefully designed scheme ensures that
the program will produces a consistent output for the same input despite
the execution paths might look differently from the attacker's perspective. To
analyze software protected under \DSVMP, the adversary is forced to use a
large number of trail runs to understand how the program algorithm works.
This significantly increases the cost of code reverse-engineering.


%Uncertainty and diversity are keys for deterring dynamic cumulative attacks.
Dynamic instruction scheduling in \DSVMP is achieved through the combination of two techniques.
Firstly, \DSVMP 
provides a rich set of bytecode handlers, which have different control flows, 
to translate a bytecode instruction to native code.
Handlers for a particular bytecode opcode  all generate an identical native machine code for the same input,
but their execution paths and data accessing patterns are different from each other.
During runtime, our VM instruction scheduler randomly selects a  bytecode handler
to translate a bytecode to the corresponding native machine code.
Since the choice of handlers is randomly determined at runtime for each bytecode
instruction and the implementation of different handlers are different,
the dynamic program execution path is likely to be different in different runs.
Secondly, \DSVMP employs a multi-VM scheme so that various code regions
can be protected using different bytecode instruction sets and VM implementations.
This further increases diversity of the program, making it even harder for an adversary to analyze
the software behavior or to reuse knowledge extracted from other software products.
 This is because different products are likely to be protected using different bytecode forms and VM implementations.


The whole is greater than the sum of the parts. These techniques, putting together,
enable \DSVMP to provide stronger code protection than any of the VM-based techniques seen so far.
We have evaluated \DSVMP on four widely used applications:  ``\texttt{md5}", ``\texttt{aescrypt}", ``\texttt{bcrypt}" and ``\texttt{gzip}". Experimental results show that \DSVMP provides stronger protection with comparable runtime overhead and code size
when compared to two commercial VM-based code obfuscation tools: Code Virtualizer~\cite{2CV} and VMProtect~\cite{3Vmprotect}.


This paper makes the following contributions:
\begin{itemize}
  \item It presents a dynamic scheduling structure for VM-based code obfuscation to protect software against dynamic cumulative attacks.
  \item It is the first to apply multiple VMs to enhance diversity of code obfuscation.
  \item It demonstrates that the proposed scheme is effective in protecting real-world software applications.
\end{itemize}

The rest of this paper is organized as follows.
Section~\ref{sec:bak} introduces the principle of classical VM-based
code obfuscation techniques and cumulative attacks scenario.
Section~\ref{sec:attack} describes the VM reverse attacking approach.
Section~\ref{sec:overview} gives an overview of \DSVMP,
which is followed by a detailed description of the design in Section~\ref{sec:dvs} and~\ref{sec:mvm}.
Section~\ref{sec:case} uses a case study to demonstrate protection scheme provided by \DSVMP.
Evaluation results are presented in Sections~\ref{sec:s-eva} and~\ref{sec:p-eva} before we discuss the related work
in Section~\ref{sec:work}. Finally, Section~\ref{sec:con} presents our work conclusions. 