\begin{abstract}
Code virtualization built upon virtual machine (VM) technologies is emerging as a viable method for implementing code obfuscation to protect programs against unauthorized analysis. State-of-the-art VM-based protection approaches use a fixed scheduling structure where the program follows a single, static execution path for the same input. Such approaches, however, are vulnerable to certain scenarios where the attacker can reuse knowledge extracted from previously seen software to crack applications using similar protection schemes. This paper presents \DSVMP, a novel VM-based code obfuscation approach for software protection. \DSVMP brings together two techniques to provide stronger code protection than prior VM-based schemes. Firstly, it uses a dynamic instruction scheduler to randomly direct the program to execute different paths without violating the correctness across different runs. By randomly choosing the program execution paths, the application exposes diverse behavior, making it much more difficult for an attacker to reuse the knowledge collected from previous runs or similar applications to perform attacks. Secondly, it employs multiple VMs to further obfuscate the relationship between VM bytecode and their interpreters, making code analysis even harder. We have implemented \DSVMP in a prototype system and evaluated it using a set of widely used applications. Experimental results show that \DSVMP provides stronger protection with comparable runtime overhead and code size when compared to two commercial VM-based code obfuscation tools.
\end{abstract}